\documentclass{aplreport}

% ====================================================================
% Document parameters  -----------------------------------------------
% ====================================================================

\makeatletter
\title{Title}
\subtitle{}
\shorttitle{\@title}
\docnumber{20XX-XX}
\author{Author}
\projectleader{\@author}{}{}{}{} %name street address phone email
\titlepagefigure{base_figures/placeholder.png} %path to title figure
\titlepagecaption{Titlepage caption.}
\titlepagecaptionfigure{} %path to caption figure
\proofreader{Proofreader}
\proofdate{20XX-XX-XX}
\dnr{20XX/XXXX/XX}
\version{0.1}
\receivercompany{Name}{}{}{}{} %name street address phone email
\receiverperson{Name}{}{}{}{} %name street address phone email
\booltrue{secret}
\keywords{Keywords}
\misc{Misc}
\makeatother

% ====================================================================
% Document macros  ---------------------------------------------------
% ====================================================================



% ====================================================================
% Document content ---------------------------------------------------
% ====================================================================
\begin{document}

% ====================================================================
\maketitlepages

% The title pages can also be made individually with
%% \makemetadata
%% \maketitle
%% \titlepageback
%% \detailspage
%% \toc


% ====================================================================
\section{Sammanfattning}
\label{sec:summary}

Sammanfattningstext.

% ====================================================================
\section{Inledning}
\label{sec:introduction}

Inledningstext.

% ====================================================================
\section{Metodik}
\label{sec:method}

Metodiktext med exempelekvation som beskriver de Broglie-våglängden för en partikel som:
%
\begin{equation}
  \lambda = \frac{h}{p}~,
  \label{eq:wavelength}
\end{equation}
%
där $h$ är Plancks konstant och $p$ är partikelns rörelsemängd.

% ====================================================================
\section{Resultat}
\label{sec:results}

Resultattext med en exempelfigur som kan refereras till som figur~\ref{fig:placeholderfigure}.

\begin{figure}[htb]
  \centering
  \includegraphics[height=4.0cm]{base_figures/placeholder.png}
  \caption{Text om exempelfiguren.}
  \label{fig:placeholderfigure}
\end{figure}

% ====================================================================
\section{Diskussion}
\label{sec:diskussion}

Diskussionstext.

% ====================================================================
\bibliographystyle{plainnat}
\bibliography{refs}

% ====================================================================
\makeappendix

\section{Appendix 1}
\label{sec:appendix1}

\end{document}
